The demographic landscape of Europe is rapidly aging, and Eurostat projects that the population aged 65 and older will reach 29.4\% by 2050 \cite{Eurostat2020}. This shift correlates with an increase in age-related cognitive impairments. Dementia cases in Europe are expected to nearly double to 18.85 million by 2050 \cite{AlzheimerEurope2019}.

This increased prevalence puts a heavy toll on healthcare and the wider community. Early identification and intervention can make a highly relevant difference in the process of cognitive decline, hence improving quality of life and reducing the burden on health care \cite{Livingston2020}. Cognitive assessment tests are paramount in screening, diagnosis, monitoring, and assessing interventions \cite{Cullen2007, Petersen2018}. However, there are several problems and issues associated with existing assessment methods: variability of administration, use of time, cultural response biases, and limitations of access \cite{Prince2013, Cordell2013, Henrich2010, Goldberg2015, Geddes2020, Bilder2020}. 

To address these challenges, this study proposes developing a web-based Mini-Mental State Examination (MMSE) application. While the MMSE is widely used to detect mild cognitive impairments, it has limitations including cultural biases \cite{Folstein1975, Tombaugh1992}. In this respect, the web-based MMSE method proposed here would aim to increase accessibility, efficiency, and reach to the MMSE test. Consequently, benefiting remote or underserved regions \cite{Bauer2012, Zygouris2017, Seifan2015}.

\subsection{The Aging Population and Cognitive Health}
Europe's demographic landscape is undergoing a significant transformation as a result of its rapidly aging population. Eurostat projections indicate that the proportion of people 65 and older will increase from 20.3\% in 2019 to 29.4\% by 2050 \cite{Eurostat2020}. This demographic change correlates with an increase in age-related cognitive issues, most notably:
\begin{enumerate}
    \item \textbf{Mild Cognitive Impairment (MCI):}
    \begin{itemize}
        \item \textit{Definition:} A condition involving slight but noticeable decline in cognitive abilities.
        \item \textit{Prevalence:} Affects 6--12\% of adults aged 60 and older \cite{Petersen2018}.
    \end{itemize}
    
    \item \textbf{Dementia:}
    \begin{itemize}
        \item Various forms are expected to rise significantly with the aging population \cite{Livingston2020}.
        \item Projected to affect nearly 19 million Europeans by 2050 \cite{AlzheimerEurope2019}.
    \end{itemize}
\end{enumerate}

These trends highlight the growing importance of cognitive health in the aging population of Europe, emphasizing the need for effective screening and intervention strategies.

The increasing prevalence of cognitive impairment poses significant challenges to healthcare systems, families, and societies. However, early detection and targeted interventions can significantly alter the course of decline. Prompt identification allows healthcare providers to implement strategies that may slow progression, improve quality of life, and alleviate overall stress on healthcare infrastructure \cite{Livingston2020}. This approach creates an opportunity for timely diagnosis, empowering patients and their families to map out future care strategies proactively. It enables them to make informed decisions about the diverse treatment options available through various channels \cite{Sperling2011}.

Considering these views, an emerging need demands efficient, accessible, and trustworthy cognitive assessment tools, enabling mass-scale screening and monitoring of the aging population based on cognitive function. From this, we have to give reasons for our research to develop a much improved Web-based Mini-Mental State Examination application to enhance accessibility and efficiency in cognitive assessment within clinical and research settings.

\subsection{ The Mini-Mental State Examination (MMSE)}

Cognitive assessment is crucial in detecting and managing cognitive impairments, especially in the elderly. These kinds of tests can check for different kinds of cognitive disorders and give an idea of how bad certain impairments are, so that doctors can see how the disease is getting worse and see what treatments are working. These tools make early detection possible and initiate timely intervention, stopping deterioration and improving quality of life, in addition to providing essential information for studies on cognitive aging \cite{Langa2015, Petersen2018, Weintraub2009}. Even though it is relevant, the administration of cognitive assessment has faced a variety of challenges: variability across settings, time-consuming paper-and-pencil methods, cultural and linguistic biases, and remote assessment-related complications \cite{Prince2013, Cordell2013, Henrich2010, Goldberg2015, Geddes2020, Bilder2020}.

Folstein et al. \cite{Folstein1975} developed the MMSE in 1975, becoming a standard cognitive screening tool in clinical practice and research. Test items cover various cognitive domains, such as orientation, attention, memory, language, and visuospatial skills of a subject \cite{Folstein1975, Tombaugh1992, Shulman2006}. In most cases, clinicians perform the test face-to-face with individuals asking them to complete tasks ranging from a subject's cognitive orientation to visual construction. The advantages of MMSE are its ease of administration and its high sensitivity toward moderate to severe levels of cognitive impairments. Nevertheless, it is known to be relatively insensitive to mild impairments, to have cultural biases, and is limited in measuring executive functions \cite{Folstein1975}.

The need to overcome traditional assessment limitations and address challenges like the COVID-19 pandemic, which highlighted the need for remote healthcare solutions, drives the move towards a web-based MMSE. A web-based MMSE can increase accessibility, reduce administration time, and enable remote patient monitoring, enhancing the ability to track changes over time \cite{Bauer2012, Seifan2015, Zygouris2017, Geddes2020, Cullum2014, Lim2020, Harrington2021}. Modern cognitive evaluations have embraced technologies, such as AI, in personalization and innovating test procedures that capture subtle changes in cognitive function. However, challenges accompany these digital versions of MMSE, including the still-existing digital divide among older adults, concerns about data security, and validation against traditional evaluation methods \cite{Bilder2020, Wild2021}. Discussion on these challenges remains open in building a robust and widely accepted web-based cognitive assessment tool.\\

\noindent\textbf{The Novelty of the Study:} This research is novel in its approach toward using state-of-the-art AI techniques with Large language models (LLMs) for cognitive assessments. In the current case, a Mixtral-Dolphin 8x7B to be run on Ollama, along with the vision model for tasks that require a visual component and fallbacks from ChatGPT v4, will be utilized. At one's disposal, this armory of technologies recently available creates new opportunities to enhance adaptability, accuracy, and interpretability in cognitive assessments. Equipped with these leading-edge AI technologies, the web-based MMSE will radically change today's traditional cognitive screening and raise a new bar across the discipline. This multi-model approach pushes the envelope on the automation of cognitive assessment, guaranteeing resilience under various cognition and system conditions domains.

\subsection{Research Objectives}

This thesis aims to develop and evaluate a web-based version of the Mini-Mental State Examination (MMSE), addressing limitations of traditional cognitive assessment methods by leveraging benefits offered by modern digital technologies. The research is guided by the following objective and key features:

\subsubsection{Primary Research Objective}

Develop and evaluate a web-based adaptation of MMSE, validating its properties against traditional cognitive assessment methods while harnessing the advantages of digital technology.

\subsubsection{Key Features and Innovations}

The web-based MMSE will incorporate the following key features and innovations:

\begin{itemize}
    \item \textbf{Adaptive Testing:} Utilize item response theory~\cite{embretson2013item} to dynamically adjust question difficulty based on user responses, enhancing sensitivity to subtle cognitive changes.
    
    \item \textbf{Multimodal Input:} Implement various input methods including touchscreen, voice recognition, and microphone interactions to accommodate diverse user capabilities.
    
    \item \textbf{Automated Scoring and Analysis:} Employ machine learning algorithms~\cite{shatte2019machine} to automate scoring processes and provide detailed cognitive performance analyses.
    
    \item \textbf{Remote Monitoring:} Develop capabilities for longitudinal tracking and remote monitoring to detect changes in cognitive function over time.
\end{itemize}

These features and innovations will guide the development and evaluation of the web-based MMSE, aiming to enhance the accessibility, efficiency, and effectiveness of cognitive assessments through innovative technology.

\subsection{Thesis Structure}

The structure takes the reader through the research process, the findings related to the web-based MMSE system, and what this tool implies. It has six main chapters related to some points about the research and what it adds regarding cognitive assessment.

\paragraph{Chapter 2: Baseline and Literature Review.} This chapter comprehensively discusses and reviews existing literature on cognitive assessments, mainly focusing on the MMSE and its adaptations. It also explores the evolution of computerized cognitive testing and discusses the role of artificial intelligence in enhancing neuropsychological assessment.

\paragraph{Chapter 3: Research Methodology.} The methodology described in this chapter encompasses the design and implementation of the web-based MMSE application. This includes the technical architecture, the development of cognitive tasks, the integration of artificial intelligence components, and the design of the user interface.

\paragraph{Chapter 4: Implementation.} This chapter presents the technical details of the application's practical implementation. It covers the system architecture, AI integration process, backend and frontend development, data management strategies, and security measures. The chapter emphasizes the thorough development process of the web-based MMSE system with AI enhancements, concentrating on the crucial design choices and technical difficulties encountered during development.

\paragraph{Chapter 5: Results and Evaluation.} This chapter outlines the implementation results and the findings from the evaluation study. It details the system's performance indicators, outcomes of various testing processes, and the comparison with conventional MMSE approaches. The chapter also analyzes the application's usability, discusses metrics for performance-related detection and monitoring of mild cognitive impairment, and examines how well the implementation met the initial objectives. Finally, it reviews the strengths and limitations of the web-based application, underlining its innovative aspects and discussing the implications for cognitive assessment practices in light of existing literature.

\paragraph{Chapter 6: Conclusion.} This final chapter summarizes the essential findings and considers the general implications of the research work. It also suggests future research directions and potential enhancements to the web-based MMSE application.\\

Each chapter builds upon the previous one to thoroughly understand the web-based cognitive assessment application's development, evaluation, and implications. This thesis aims to contribute significantly to the field by improving the accessibility, efficiency, and effectiveness of cognitive assessments through innovative technology.