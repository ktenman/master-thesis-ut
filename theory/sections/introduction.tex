Europe is experiencing a rapid demographic shift towards an aging population. Eurostat projections indicate that by 2050, 29.4\% of the European population will be 65 years or older \cite{Eurostat2020}. This aging trend correlates with a rise in age-related cognitive impairments, with dementia cases in Europe expected to double to 18.85 million by 2050 nearly \cite{AlzheimerEurope2019}.

This increased prevalence of cognitive impairments places a heavy burden on healthcare systems and the wider community. Early detection and intervention can significantly impact the progression of cognitive decline, potentially improving quality of life and reducing healthcare costs \cite{Livingston2020}. In this context, cognitive assessment tests are crucial in screening, diagnosis, monitoring, and evaluating interventions \cite{Cullen2007, Petersen2018}.

This study proposes developing a web-based Mini-Mental State Examination (MMSE) application to address these challenges. While the MMSE is widely used to detect mild cognitive impairments, it has limitations including cultural biases \cite{Folstein1975, Tombaugh1992}. In this respect, the web-based MMSE method proposed here would aim to increase accessibility, efficiency, and reach of the MMSE test, consequently benefiting remote or underserved regions \cite{Bauer2012, Zygouris2017, Seifan2015}.

\subsection{The Aging Population and Cognitive Health}
Europe's demographic landscape is undergoing a significant transformation due to its rapidly aging population, a change that correlates with an increase in age-related cognitive issues, most notably:

\begin{enumerate}
    \item \textbf{Mild Cognitive Impairment (MCI):}
    \begin{itemize}
        \item \textit{Definition:} A condition involving slight but noticeable decline in cognitive abilities.
        \item \textit{Prevalence:} Affects 6--12\% of adults aged 60 and older \cite{Petersen2018}.
    \end{itemize}
    
    \item \textbf{Dementia:}
    \begin{itemize}
        \item \textit{Definition:} A syndrome characterized by a significant decline in cognitive function, affecting memory, thinking, orientation, comprehension, calculation, learning capacity, language, and judgment.  
        \item Various forms of dementia are expected to rise significantly with the aging population \cite{Livingston2020}. 
        \item The number of Europeans with dementia is projected to increase substantially in the coming decades \cite{AlzheimerEurope2019}.
    \end{itemize}
\end{enumerate}

These trends highlight the growing importance of cognitive health in the aging population of Europe, emphasizing the need for effective screening and intervention strategies.

The increasing prevalence of cognitive impairment poses significant challenges to healthcare systems, families, and societies. However, early detection and targeted interventions can significantly alter the course of decline. Prompt identification allows healthcare providers to implement strategies that may slow progression, improve quality of life, and alleviate overall stress on healthcare infrastructure \cite{Livingston2020}. This approach creates an opportunity for timely diagnosis, empowering patients and their families to map out future care strategies proactively. It enables them to make informed decisions about the diverse treatment options available through various channels \cite{Sperling2011}.

Considering these views, an emerging need demands efficient, accessible, and trustworthy cognitive assessment tools, enabling mass-scale screening and monitoring of the aging population based on cognitive function. This research necessitates the development of an improved web-based Mini-Mental State Examination application to enhance accessibility and efficiency in cognitive assessment within clinical and research settings.

\subsection{The Mini-Mental State Examination (MMSE)}

Cognitive assessment is crucial in detecting and managing cognitive impairments, especially in the elderly. These kinds of tests can check for different kinds of cognitive disorders and give an idea of how bad certain impairments are, so that doctors can see how the disease is getting worse and see what treatments are working. These tools make early detection possible and initiate timely intervention, stopping deterioration and improving quality of life, in addition to providing essential information for studies on cognitive aging \cite{Langa2015, Petersen2018, Weintraub2009}. Even though it is relevant, the administration of cognitive assessment has faced a variety of challenges: variability across settings, time-consuming paper-and-pencil methods, cultural and linguistic biases, and remote assessment-related complications \cite{Prince2013, Cordell2013, Henrich2010, Goldberg2015, Geddes2020, Bilder2020}.

Folstein et al. \cite{Folstein1975} developed the MMSE in 1975, becoming a standard cognitive screening tool in clinical practice and research. Test items cover various cognitive domains, such as orientation, attention, memory, language, and visuospatial skills of a subject \cite{Folstein1975, Tombaugh1992, Shulman2006}. In most cases, clinicians perform the test face-to-face with individuals asking them to complete tasks ranging from a subject's cognitive orientation to visual construction. The advantages of MMSE are its ease of administration and its high sensitivity toward moderate to severe levels of cognitive impairments. Nevertheless, it is known to be relatively insensitive to mild impairments, to have cultural biases, and is limited in measuring executive functions \cite{Folstein1975}.

The need to overcome traditional assessment limitations and address challenges like the COVID-19 pandemic, which highlighted the need for remote healthcare solutions, drives the move towards a web-based MMSE. A web-based MMSE can increase accessibility, reduce administration time, and enable remote patient monitoring, enhancing the ability to track changes over time \cite{Bauer2012, Seifan2015, Zygouris2017, Geddes2020, Cullum2014, Lim2020, Harrington2021}. Modern cognitive evaluations have embraced technologies, such as AI, in personalization and innovating test procedures that capture subtle changes in cognitive function. However, challenges accompany these digital versions of MMSE, including the still-existing digital divide among older adults, concerns about data security, and validation against traditional evaluation methods \cite{Bilder2020, Wild2021}. Discussing these challenges remains open to building a robust and widely accepted web-based cognitive assessment tool.\\

\noindent\textbf{The Novelty of the Study:} This research introduces a novel approach to cognitive assessments by employing state-of-the-art AI techniques with Large Language Models (LLMs). Specifically, the study utilizes the Llama 3.1:70B model, executed on the Ollama platform, in conjunction with a vision model for tasks requiring visual processing. Additionally, fallback mechanisms from ChatGPT 4o are integrated to ensure robustness across diverse assessment scenarios. This advanced suite of technologies enhances cognitive assessments' adaptability, accuracy, and interpretability, setting a new standard in the field. Adding these advanced AI technologies to the web-based MMSE is meant to improve traditional ways of testing cognitive abilities. This will push the limits of automation and ensure the test works well in a wide range of cognitive and system conditions.

\subsection{Research Objectives}

This thesis aims to develop and evaluate a web-based version of the Mini-Mental State Examination (MMSE), addressing the limitations of traditional cognitive assessment methods by leveraging the benefits offered by modern digital technologies. The research focuses on enhancing accessibility, efficiency, and accuracy in cognitive assessments, particularly for remote or underserved populations.

The primary research objective is to develop and evaluate a web-based adaptation of the MMSE, validating its properties against traditional cognitive assessment methods while harnessing the advantages of digital technology.

To achieve this, the web-based MMSE will incorporate the following key features and innovations:

\begin{itemize}
    \item \textbf{Adaptive Testing:} Utilize item response theory~\cite{embretson2013item} to dynamically adjust question difficulty based on user responses, enhancing sensitivity to subtle cognitive changes.
    
    \item \textbf{Multimodal Input:} Implement various input methods including touchscreen, voice recognition, and microphone interactions to accommodate diverse user capabilities.
    
    \item \textbf{Automated Scoring and Analysis:} Employ machine learning algorithms~\cite{shatte2019machine} to automate scoring processes and provide detailed cognitive performance analyses.
    
    \item \textbf{Remote Monitoring:} Develop capabilities for longitudinal tracking and remote monitoring to detect changes in cognitive function over time.
\end{itemize}

These features and innovations will guide the development and evaluation of the web-based MMSE, aiming to enhance the accessibility, efficiency, and effectiveness of cognitive assessments through innovative technology.

\subsection{Thesis Structure}

The structure takes the reader through the research process, the findings related to the web-based MMSE system, and what this tool implies. It has six main chapters related to some points about the research and what it adds regarding cognitive assessment.

\paragraph{Chapter 2: Baseline and Literature Review.} This chapter comprehensively discusses and reviews existing literature on cognitive assessments, mainly focusing on the MMSE and its adaptations. It also explores the evolution of computerized cognitive testing and discusses the role of artificial intelligence in enhancing neuropsychological assessment.

\paragraph{Chapter 3: Research Methodology.} The methodology described in this chapter encompasses the design and implementation of the web-based MMSE application. This includes the technical architecture, the development of cognitive tasks, the integration of artificial intelligence components, and the design of the user interface.

\paragraph{Chapter 4: Implementation.} This chapter presents the technical details of the application's practical implementation. It covers the system architecture, AI integration process, backend and frontend development, data management strategies, and security measures. The chapter emphasizes the thorough development process of the web-based MMSE system with AI enhancements, concentrating on the crucial design choices and technical difficulties encountered during development.

\paragraph{Chapter 5: Results and Evaluation.} This chapter outlines the implementation results and the findings from the evaluation study. It details the system's performance indicators, outcomes of various testing processes, and the comparison with conventional MMSE approaches. The chapter also analyzes the application's usability, discusses metrics for performance-related detection and monitoring of mild cognitive impairment, and examines how well the implementation met the initial objectives. Finally, it reviews the strengths and limitations of the web-based application, underlining its innovative aspects and discussing the implications for cognitive assessment practices in light of existing literature.

\paragraph{Chapter 6: Conclusion.} This final chapter summarizes the essential findings and considers the general implications of the research work. It also suggests future research directions and potential enhancements to the web-based MMSE application.\\

Each chapter builds upon the previous one to thoroughly understand the web-based cognitive assessment application's development, evaluation, and implications. This thesis aims to contribute significantly to the field by improving the accessibility, efficiency, and effectiveness of cognitive assessments through innovative technology.