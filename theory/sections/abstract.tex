
% If the thesis is printed on both sides of the page then 
% the second page must be must be empty. Comment this out
% if you print only to one side of the page comment this out
%\newpage
%\thispagestyle{empty}    
%\phantom{Text to fill the page}
% END OF EXTRA PAGE WITHOUT NUMBER


%===COMPULSORY INFO PAGE
\newpage

%=== Info in English
\newcommand\EngInfo{{%
\selectlanguage{english}
\noindent\textbf{\large \thesistitle{}}

\vspace*{1ex}

\noindent\textbf{Abstract:}

\noindent
The growing prevalence of dementia among older adults underscores the need for efficient and accessible cognitive assessment tools. Traditional methods, such as the Mini-Mental State Examination (MMSE), are typically conducted in clinical settings using paper-based formats, limiting access due to resource constraints and needing trained professionals. This thesis addresses these challenges by converting the MMSE into an AI-powered, web-based application, allowing assessments to be completed at home with minimal non-professional assistance.

The digital implementation leverages sophisticated artificial intelligence (AI) models, particularly the Llama 3.1:70B, for automating the administration of the MMSE. This makes it more consistent and sensitive to small changes in cognitive function. By leveraging Machine Learning (ML) and Natural Language Processing (NLP), the system improves the consistency, accuracy, and accessibility of cognitive assessments through web-based administration.

Following the Design Science Research (DSR) framework, the research integrates modern web technologies with a hybrid AI approach, optimizing performance and ensuring data privacy. In trials, the AI-powered MMSE achieved a 92.9\% success rate in confirming response correctness compared to traditional methods and slightly higher user satisfaction despite longer administration times.

While this work significantly improves cognitive assessment accessibility and sensitivity, further studies are needed to validate its effectiveness across diverse clinical settings. Future research should optimize response times, expand language support, and address ethical considerations in AI-driven cognitive assessments.
\vspace*{1ex}

\noindent\textbf{Keywords:} cognitive assessment, dementia, web-based MMSE, artificial intelligence, machine learning, natural language processing, accessibility, healthcare democratization

\vspace*{1ex}

\noindent\textbf{CERCS:} P170 - Computer science, numerical analysis, systems, control

\vspace*{1ex}
}}%\newcommand\EngInfo


%=== Info in Estonian
\newcommand\EstInfo{{%
\newpage
\selectlanguage{estonian}
\noindent\textbf{\large \thesistitleET}
\vspace*{1ex}

\noindent\textbf{Lühikokkuvõte:} 

\noindent
Eakate seas kasvav dementsuse levik rõhutab vajadust tõhusate ja kättesaadavate kognitiivsete hindamisvahendite järele. Traditsioonilised meetodid, nagu Mini-Mental State Examination (MMSE), viiakse tavaliselt läbi paberil, kliinilises keskkonnas, piirates juurdepääsu ressursside nappuse ja koolitatud spetsialistide vajaduse tõttu. See uurimustöö tegeleb nende väljakutsetega, muutes MMSE AI-toega veebirakenduseks, võimaldades koduseid hindamisi vähese abiga.

Näiteks kasutab digitaalne MMSE versioon täiustatud AI-mudeleid, nagu Llama 3.1:70B, et automatiseerida testimisprotsessi. See muudab hindamise järjepidevamaks ja tundlikumaks kognitiivse funktsiooni väikeste muutuste suhtes. Masinõppe (ML) ja loomuliku keele töötlemise (NLP) abil parandab süsteem kognitiivsete hindamiste järjepidevust, täpsust ja kättesaadavust, viies need läbi veebis.

Disainiteaduse uurimisraamistiku (DSR) järgimisel ühendab uurimustöö kaasaegsed veebitehnoloogiad hübriidse AI-lähenemisega, optimeerides jõudlust ja tagades andmete privaatsuse. Katsetes saavutas AI-toega MMSE 92.9\% edukuse määra vastuste õigsuse kinnitamisel võrreldes traditsiooniliste meetoditega ning veidi kõrgema kasutajate rahulolu, kuigi testide läbiviimise aeg oli pikem.

Kuigi see uurimustöö parandab oluliselt kognitiivsete hindamiste kättesaadavust ja tundlikkust, on vaja täiendavaid uuringuid, et kinnitada selle tõhusust erinevates kliinilistes tingimustes. Tulevased uuringud peaksid optimeerima vastuste kiirust, laiendama keelelist tuge ning tegelema AI-põhiste kognitiivsete hindamiste eetiliste küsimustega.
\vspace*{1ex}

\noindent\textbf{Võtmesõnad:} kognitiivne hindamine, dementsus, veebipõhine MMSE, tehisintellekt, masinõpe, loomuliku keele töötlus, kättesaadavus, tervishoiuteenuste demokratiseerimine

\vspace*{1ex}

\noindent\textbf{CERCS:} P170 - Arvutiteadus, arvanalüüs, süsteemid, kontroll 

\vspace*{1ex}
}}%\newcommand\EstInfo


%=== Determine the order of languages on Info page
\iflanguage{english}{\EngInfo}{\EstInfo}
\iflanguage{estonian}{\EngInfo}{\EstInfo}