
% If the thesis is printed on both sides of the page then 
% the second page must be must be empty. Comment this out
% if you print only to one side of the page comment this out
%\newpage
%\thispagestyle{empty}    
%\phantom{Text to fill the page}
% END OF EXTRA PAGE WITHOUT NUMBER


%===COMPULSORY INFO PAGE
\newpage

%=== Info in English
\newcommand\EngInfo{{%
\selectlanguage{english}
\noindent\textbf{\large \thesistitle{}}

\vspace*{1ex}

\noindent\textbf{Abstract:}

\noindent
The increasing prevalence of dementia among older adults highlights the need for efficient and accessible cognitive assessment tools. Traditional assessments like the Mini-Mental State Examination (MMSE) are typically conducted in specialized clinics, limiting access due to resource constraints and needing trained professionals. This thesis addresses these challenges by developing and evaluating an AI-powered, web-based MMSE application that enables individuals to complete assessments at home with minimal assistance from non-professionals.

Using Machine Learning (ML) and Natural Language Processing (NLP), the study automates the MMSE testing process to improve accessibility and efficiency. Advanced AI models, including Llama 3.1:70b and ChatGPT 4o, address limitations in traditional MMSE methods, such as standardization issues and sensitivity to subtle cognitive changes.

Following the Design Science Research (DSR) framework, the system integrates modern web technologies and a hybrid AI approach to optimize performance and data privacy. The AI-powered MMSE, utilizing models such as Llama 3.1, demonstrated a 92.9\% success rate in confirming response correctness compared to traditional paper-based methods. Additionally, the web-based MMSE achieved slightly higher user satisfaction scores despite longer administration times.

This research contributes to the field by enhancing the consistency, accessibility, and sensitivity of cognitive assessments. While promising, further studies are needed to validate this approach in diverse clinical settings. Future research should optimize response times, expand language support, and address ethical implications in AI-driven cognitive assessments.
\vspace*{1ex}

\noindent\textbf{Keywords:} cognitive assessment, dementia, web-based MMSE, artificial intelligence, machine learning, natural language processing, accessibility, healthcare democratization

\vspace*{1ex}

\noindent\textbf{CERCS:} P170 - Computer science, numerical analysis, systems, control

\vspace*{1ex}
}}%\newcommand\EngInfo


%=== Info in Estonian
\newcommand\EstInfo{{%
\newpage
\selectlanguage{estonian}
\noindent\textbf{\large \thesistitleET}
\vspace*{1ex}

\noindent\textbf{Lühikokkuvõte:} 

\noindent
Vanemaealiste seas on dementsus sagedasem. See rõhutab tõhusate ja kättesaadavate kognitiivsete hindamisvahendite vajadust. Traditsioonilised hindamismeetodid, nagu Mini-Mental State Examination (MMSE), viiakse tavaliselt läbi spetsialiseeritud kliinikutes ja asutustes. See piirab juurdepääsu, kuna selleks on vaja nii ressursse kui ka koolitatud spetsialiste. Käesolev lõputöö käsitleb neid väljakutseid. Töö raames arendatakse ja hinnatakse tehisintellektil põhinevat veebipõhist MMSE-rakendust, mis võimaldab inimestel kodus hindamisi sooritada minimaalse abiga, mis ei vaja erialaseid oskusi.

Masinõppe (ML) ja loomuliku keele töötlemise (NLP) abil automatiseeritakse MMSE testimisprotsess, mis suurendab juurdepääsu ja tõhustab protsessi. Täiustatud tehisintellekti mudelid, nagu Llama 3.1:70b ja ChatGPT 4o, aitavad ületada traditsiooniliste MMSE meetodite piiranguid, näiteks standardiseerimisprobleeme ja tundlikkust peenete kognitiivsete muutuste suhtes.

Disainiteaduse uurimismeetodi (DSR) raames integreerib rakendus kaasaegseid veebitehnoloogiaid ning hübriidset tehisintellekti lähenemist, et optimeerida jõudlust ja andmete privaatsust. Tehisintellektil põhinev MMSE, kasutades mudeleid nagu Llama 3.1, saavutas 92.9\% vastuste korrektsuse, ületades traditsioonilised paberil põhinevad meetodid. Vaatamata sellele, et veebipõhise MMSE sooritamine võtab rohkem aega, saavutas see siiski veidi kõrgemad kasutajate rahulolu hinnangud.

See uurimustöö annab olulise panuse valdkonda, parandades kognitiivsete hindamiste järjepidevust, juurdepääsu ja tundlikkust. Kuigi lähenemine on paljulubav, vajab see siiski täiendavaid uuringuid, et kinnitada selle tõhusust mitmekesistes kliinilistes tingimustes. Tulevased uuringud peaksid keskenduma vastamisaegade optimeerimisele, keelelise toe laiendamisele ja tehisintellekti juhitud kognitiivsete hindamiste eetiliste aspektide käsitlemisele.
\vspace*{1ex}

\noindent\textbf{Võtmesõnad:} kognitiivne hindamine, dementsus, veebipõhine MMSE, tehisintellekt, masinõpe, loomuliku keele töötlus, kättesaadavus, tervishoiuteenuste demokratiseerimine

\vspace*{1ex}

\noindent\textbf{CERCS:} P170 - Arvutiteadus, arvanalüüs, süsteemid, kontroll 

\vspace*{1ex}
}}%\newcommand\EstInfo


%=== Determine the order of languages on Info page
\iflanguage{english}{\EngInfo}{\EstInfo}
\iflanguage{estonian}{\EngInfo}{\EstInfo}