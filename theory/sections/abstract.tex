
% If the thesis is printed on both sides of the page then 
% the second page must be must be empty. Comment this out
% if you print only to one side of the page comment this out
%\newpage
%\thispagestyle{empty}    
%\phantom{Text to fill the page}
% END OF EXTRA PAGE WITHOUT NUMBER


%===COMPULSORY INFO PAGE
\newpage

%=== Info in English
\newcommand\EngInfo{{%
\selectlanguage{english}
\noindent\textbf{\large \thesistitle{}}

\vspace*{1ex}

\noindent\textbf{Abstract:}

\noindent

Together with the aging population, cognitive impairments related to aging, in particular, mild cognitive impairment and various kinds of dementia, have been becoming more common in Europe \cite{Eurostat2020,Livingston2020}. Thus, the early identification and follow-up of cognitive decline need to provide timely interventions for improved patient outcomes \cite{Petersen2018}. The Mini-Mental State Examination is one of the most commonly used tools for evaluating cognition, but its classical administration has certain limitations regarding accessibility, efficiency, and standardization.

This thesis presents the development and evaluation of a novel web-based MMSE system that aims to resolve these challenges within the context of the COVID-19 pandemic. Our system will strive to improve the access, efficiency, and remote administration of cognitive testing by using cutting-edge web technologies and artificial intelligence. In this respect, this study mainly focuses on:

\begin{itemize}
    \item Design and Implementation of Mobile Friendly/Responsive Web-Based Application for MMSE Administration: Vue.js, Spring Boot.
    \item Artificial intelligence techniques enable the system to automatically score and analyze the MMSE, integrating with natural language processing and computer vision.
    \item Testing the system for its validity, reliability, and user experience with the traditional methods of administering MMSE.
    \item Ethical considerations include enabling this AI-augmented duty to be bias-free and culturally sensitive while protecting data privacy.
\end{itemize}

The following methodology is structured to include a literature review, system design and development, pilot testing, and the final evaluation study. Results: Web-based cognitive assessment tools have the potential to enhance early detection capabilities, improve access to cognitive screening, and facilitate large-scale data collection for research purposes \cite{Sperling2011}.

This interdisciplinary project makes valuable contributions to computer science, psychology, and healthcare by providing proof of concept for technology-driven solutions to increasingly essential challenges to the cognitive health of aging populations. The result has significant implications for clinical practices, management research, and public health initiatives toward managing the decline of cognitive abilities among the elderly, particularly in resource-poor or remotely located settings.

\vspace*{1ex}
\noindent\textbf{Keywords:} cognitive assessment, MMSE, web-based application, artificial intelligence, e-health, natural language processing, speech recognition, Spring Boot, Vue.js, Java, JHipster, ChatGPT, Dolphin Mixtral, Ollama, Llama, sentiment analysis, grammatical correctness, Postgres, MinIO, TensorFlow, PyTorch, Flask

\vspace*{1ex}

\noindent\textbf{CERCS:}

\begin{itemize}
    \item P175 - Informatics, systems theory
    \item B110 - Bioinformatics, medical informatics, biomathematics, biometrics
\end{itemize}

\vspace*{1ex}

\end{abstract}
}}%\newcommand\EngInfo


%=== Info in Estonian
\newcommand\EstInfo{{%
\newpage
\selectlanguage{estonian}
\noindent\textbf{\large \thesistitleET}
\vspace*{1ex}

\noindent\textbf{Lühikokkuvõte:} 

\noindent

Euroopa rahvastiku vananemisega kaasneb vananemisega seotud kognitiivsete häirete, eriti kerge kognitiivse häire ja erinevate dementsuse vormide sagenemine \cite{Eurostat2020,Livingston2020}. Seetõttu on üha olulisem kognitiivse languse varajane tuvastamine ja jälgimine, et võimaldada õigeaegseid sekkumisi ja parandada patsientide elukvaliteeti \cite{Petersen2018}. Vaimse Seisundi Miniuuring (MMSE) on laialdaselt kasutatav vahend kognitiivsete võimete hindamiseks, kuid selle traditsiooniline läbiviimine on piiratud ligipääsetavuse, tõhususe ja standardiseerimise osas.

Käesolev magistritöö tutvustab uudset veebipõhist MMSE süsteemi, mis on loodud nende väljakutsete lahendamiseks, eriti COVID-19 pandeemia kontekstis. Meie süsteem kasutab kaasaegseid veebitehnoloogiaid ja tehisintellekti, et parandada kognitiivsete testide kättesaadavust, tõhusust ja kaugadministreerimist. Uurimistöö keskendub järgmistele aspektidele:

\begin{itemize}
    \item Mobiilisõbraliku ja kohanduva veebipõhise MMSE rakenduse loomine, kasutades Vue.js ja Spring Boot tehnoloogiaid.
    \item Tehisintellekti rakendamine MMSE automaatseks hindamiseks ja analüüsimiseks, integreerides loomuliku keele töötlemise ja arvutinägemise.
    \item Süsteemi põhjalik testimine, võrreldes selle kehtivust, usaldusväärsust ja kasutajakogemust MMSE traditsiooniliste läbiviimismeetoditega.
    \item Eetiliste aspektide käsitlemine, tagades tehisintellekti erapooletuse ja kultuurilise tundlikkuse ning kaitstes samal ajal andmete privaatsust.
\end{itemize}

Uurimistöö metoodika hõlmab põhjalikku kirjanduse ülevaadet, süsteemi disaini ja arendust, pilootkatsetamist ning lõplikku hindamisuuringut. Tulemused näitavad, et veebipõhistel kognitiivse hindamise vahenditel on suur potentsiaal parandada varajast avastamist, laiendada juurdepääsu kognitiivsele sõeluuringule ja hõlbustada suuremahuliste andmete kogumist teadustöö tarbeks \cite{Sperling2011}.

See interdistsiplinaarne projekt annab olulise panuse arvutiteadusesse, psühholoogiasse ja tervishoidu, pakkudes innovaatilist tehnoloogiapõhist lahendust vananeva elanikkonna kognitiivse tervise väljakutsetele. Uurimistöö tulemused on märkimisväärsed kliinilise praktika, tervishoiujuhtimise ja rahvatervise algatuste jaoks, eriti eakate kognitiivse languse haldamisel ressursivaestes või kaugetes piirkondades.


\vspace*{1ex}
\noindent\textbf{Võtmesõnad:} kognitiivne hindamine, MMSE, veebipõhine rakendus, tehisintellekt, e-tervis, loomuliku keele töötlemine, kõnetuvastus, Spring Boot, Vue.js, Java, JHipster, ChatGPT, Dolphin Mixtral, Ollama, Llama, sentimentanalüüs, grammatiline korrektsus, Postgres, MinIO, TensorFlow, PyTorch, Flask

\vspace*{1ex}
\newpage
\noindent\textbf{CERCS:} 
\begin{itemize}
    \item P175 - Informaatika, süsteemiteooria
    \item B110 - Bioinformaatika, meditsiiniinformaatika, biomatemaatika, biomeetrika
\end{itemize}
\vspace*{1ex}
}}%\newcommand\EstInfo


%=== Determine the order of languages on Info page
\iflanguage{english}{\EngInfo}{\EstInfo}
\iflanguage{estonian}{\EngInfo}{\EstInfo}

\newpage
\tableofcontents