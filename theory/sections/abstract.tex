
% If the thesis is printed on both sides of the page then 
% the second page must be must be empty. Comment this out
% if you print only to one side of the page comment this out
%\newpage
%\thispagestyle{empty}    
%\phantom{Text to fill the page}
% END OF EXTRA PAGE WITHOUT NUMBER


%===COMPULSORY INFO PAGE
\newpage

%=== Info in English
\newcommand\EngInfo{{%
\selectlanguage{english}
\noindent\textbf{\large \thesistitle{}}

\vspace*{1ex}

\noindent\textbf{Abstract:}

\noindent
The increasing incidence of dementia among the elderly highlights the critical demand for cognitive assessment tools that are both efficient and widely accessible. Traditional methods, such as the Mini-Mental State Examination (MMSE), are typically conducted in clinical settings using paper-based formats, limiting access due to resource constraints and needing trained professionals. This thesis addresses these challenges by converting the MMSE into an AI-powered, web-based application, allowing assessments to be completed at home with minimal non-professional assistance.

The digital implementation leverages sophisticated artificial intelligence (AI) models, particularly the Llama 3.1:70B, for automating the administration of the MMSE. This makes it more consistent and sensitive to small changes in cognitive function. By leveraging Machine Learning (ML) and Natural Language Processing (NLP), the system improves the consistency, accuracy, and accessibility of cognitive assessments through web-based administration.

Adopting the Design Science Research (DSR) framework, this study incorporates contemporary web technologies alongside a hybrid AI strategy, enhancing performance while safeguarding data privacy. In trials, the AI-powered MMSE achieved a 92.9\% success rate in confirming response correctness compared to traditional methods and slightly higher user satisfaction despite longer administration times.

While this work significantly improves cognitive assessment accessibility and sensitivity, further studies are needed to validate its effectiveness across diverse clinical settings. Future research should optimize response times, expand language support, and address ethical considerations in AI-driven cognitive assessments.
\vspace*{1ex}

\noindent\textbf{Keywords:} cognitive assessment, dementia, web-based MMSE, artificial intelligence, machine learning, natural language processing, accessibility, healthcare democratization

\vspace*{1ex}

\noindent\textbf{CERCS:} P170 - Computer science, numerical analysis, systems, control

\vspace*{1ex}
}}%\newcommand\EngInfo


%=== Info in Estonian
\newcommand\EstInfo{{%
\newpage
\selectlanguage{estonian}
\noindent\textbf{\large \thesistitleET}
\vspace*{1ex}

\noindent\textbf{Lühikokkuvõte:} 

\noindent
Eakate seas dementsuse esinemissageduse kasv rõhutab tõhusate ja laialdaselt kättesaadavate kognitiivse hindamise tööriistade vajadust. Traditsioonilised meetodid, nagu Mini-Mental State Examination (MMSE), viiakse tavaliselt läbi kliinilistes tingimustes, kasutades paberkandjal vorme, mis piirab nende kättesaadavust ressursside nappuse ja koolitatud spetsialistide vajaduse tõttu. Käesolev lõputöö tegeleb nende väljakutsetega, muutes MMSE tehisintellektipõhiseks veebirakenduseks, mis võimaldab hinnangu andmist kodustes tingimustes minimaalse mitteprofessionaalse abiga.

Magistritöö käigus loodud digitaalne prototüüp kasutab keerukaid tehisintellekti (AI) mudeleid, nagu Llama 3.1 70B, et MMSE läbiviimist automatiseerida. See muudab hindamise järjepidevamaks ja tundlikumaks väikeste kognitiivsete funktsioonide muutuste suhtes. Masinõppe (ML) ja loomuliku keele töötlemise (NLP) abil parandab süsteem kognitiivsete hinnangute järjepidevust, täpsust ja kättesaadavust veebipõhise läbiviimise kaudu.

Uurimistöös on kasutatud disainiteaduse uurimismeetodit (ingl. Design Science Research, DSR), mis ühendab kaasaegsed veebitehnoloogiad ja hübriidse AI-strateegia, et parandada jõudlust ja kaitsta andmete privaatsust. Katsetes saavutas AI-põhine MMSE 92.9\% täpsuse vastuste õigsuse tuvastamisel võrreldes traditsiooniliste meetoditega. Kuigi läbiviimise aeg oli veidi pikem, oli kasutajate rahulolu siiski pisut suurem.

See töö annab märkimisväärse panuse kognitiivse hindamise kättesaadavuse ja tundlikkuse parandamisse, kuid edasised uuringud on vajalikud, et kinnitada selle efektiivsust erinevates kliinilistes tingimustes. Tulevased uuringud peaksid keskenduma hindamiskiiruse optimeerimisele, keelelise toe laiendamisele ning AI-põhiste kognitiivsete mõõtmiste eetiliste küsimuste käsitlemisele.
\vspace*{1ex}

\noindent\textbf{Võtmesõnad:} kognitiivne hindamine, dementsus, veebipõhine MMSE, tehisintellekt, masinõpe, loomuliku keele töötlus, kättesaadavus, tervishoiuteenuste demokratiseerimine

\vspace*{1ex}

\noindent\textbf{CERCS:} P170 - Arvutiteadus, arvanalüüs, süsteemid, kontroll 

\vspace*{1ex}
}}%\newcommand\EstInfo


%=== Determine the order of languages on Info page
\iflanguage{english}{\EngInfo}{\EstInfo}
\iflanguage{estonian}{\EngInfo}{\EstInfo}