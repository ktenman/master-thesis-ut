
% If the thesis is printed on both sides of the page then 
% the second page must be must be empty. Comment this out
% if you print only to one side of the page comment this out
%\newpage
%\thispagestyle{empty}    
%\phantom{Text to fill the page}
% END OF EXTRA PAGE WITHOUT NUMBER


%===COMPULSORY INFO PAGE
\newpage

%=== Info in English
\newcommand\EngInfo{{%
\selectlanguage{english}
\noindent\textbf{\large \thesistitle{}}

\vspace*{1ex}

\noindent\textbf{Abstract:}

\noindent
In a world where the specter of dementia looms large, early detection is our best defense. But traditional cognitive assessments like the Mini-Mental State Examination (MMSE) are often out of reach for those who need them most, hindered by access barriers, resource constraints, and the need for trained professionals [1]. It's a problem that calls for a bold solution - and that's where this thesis comes in.

We present a groundbreaking web-based MMSE application that harnesses the power of artificial intelligence to make cognitive screening more accessible and efficient than ever before. By leveraging machine learning and natural language processing [2], our system automates the entire testing process from start to finish, opening the door to underserved populations and easing the burden on overtaxed healthcare systems.

But we didn't stop at just building the tool - we put it through its paces with rigorous validation testing, pitting its performance against the tried-and-true methods of the past [3]. The results speak for themselves: our AI-powered MMSE isn't just a feasible alternative to traditional testing, it's a powerful force for change in the fight against cognitive decline.

This research is more than just a technological breakthrough - it's a vision of a future where cutting-edge innovation meets real-world impact. By democratizing access to vital healthcare services [4], we're not just improving outcomes for individual patients - we're paving the way for a healthier, more resilient society as a whole. It's a testament to the transformative potential of technology, and a rallying cry for a world that refuses to let the challenges of aging stand in the way of living life to the fullest.
\vspace*{1ex}

\noindent\textbf{Keywords:} cognitive assessment, dementia, web-based MMSE, artificial intelligence, machine learning, natural language processing, accessibility, healthcare democratization

\vspace*{1ex}

\noindent\textbf{CERCS:} P170 - Computer science, numerical analysis, systems, control

\vspace*{1ex}
}}%\newcommand\EngInfo


%=== Info in Estonian
\newcommand\EstInfo{{%
\newpage
\selectlanguage{estonian}
\noindent\textbf{\large Detsentraliseeritud asukoha tõendamise suunas}
\vspace*{1ex}

\noindent\textbf{Lühikokkuvõte:} 

\noindent

Asukohapõhised teenused on tänapäeva ühiskonnas üldlevinud ning nende integreeritavus erinevate rakenduste ja tehnoloogiatega on kujundanud meie suhtlust füüsilise maailmaga. Asukohapõhiste süsteemide praegune seisund ei taga kaugeltki asukohateabe terviklikkust, eriti usaldamatutes keskkondades, kus puuduvad individuaalsed usaldusväärsuse tagatised. On vaja paradigma muutust, et tagada turvalisus geograafilise manipuleerimise või asukoha võltsimise vastu. Nende nõuetele vastamiseks võivad digitaalsed ja tõendatavad asukohapäringu süsteemid aidata realiseerida asukohapõhist autentimist või autoriseerimist vaenulikes keskkondades. Sellistele süsteemidele leidub laialdaselt rakendusi nutikate linnade, laiendatud demokraatia, digitaalse terviklikkuse, vastutuse ja interneti läbipaistvuse valdkondades. Käesolevas lõputöös tutvustame probleemile uudset lähenemist, uurides asukohatõestussüsteemide paradigmat ja tuginedes olemasolevatele süsteemidele, et liikuda täielikult detsentraliseeritud asukohatõenduse poole. Kasutades võrgustumistehnoloogiad ja lubadeta konsensusmehhanisme, määratleme uue protokolli ning rakendame ja hindame kontseptsioonitõestust, mis tutvustab täielike, kontrollitavate ning ruumilis-ajaliselt kindlaid asukoha tõendite genereerimist.

\vspace*{1ex}

\noindent\textbf{Võtmesõnad:} asukohapõhised teenused, asukohatõend, võrgustumistehnoloogiad, lubadeta konsensus, plokiahel, targad lepingud

\vspace*{1ex}

\noindent\textbf{CERCS:} P170 - Arvutiteadus, arvanalüüs, süsteemid, kontroll 

\vspace*{1ex}
}}%\newcommand\EstInfo


%=== Determine the order of languages on Info page
\iflanguage{english}{\EngInfo}{\EstInfo}
\iflanguage{estonian}{\EngInfo}{\EstInfo}

\newpage
\tableofcontents