TODO

\subsection*{Acknowledgments}

The author is deeply indebted to all the people and sources contributing to this thesis. Valuable mentorship, expertise, and support by Dr. Mohamad Gharib kept the author focused during his research journey. This work has gained much from his insight. The University of Tartu, especially the Institute of Computer Science, provides an excellent academic environment and crucial resources for this work. The author thanks Eduardo Brito for generously sharing his ideas and providing his LaTeX template.

Several technological tools facilitate thesis development. LaTeX serves as the base of the writing environment, while Visual Studio Code is the primary development platform. The author develops writing and collaboration using the LaTeX Workshop plugin and Overleaf. He employs Git and GitHub for version control, hosting LaTeX source files and the MMSE prototype app code, as detailed in Appendix~\ref{appendix:writing-workflow}.

AI assistants aid the writing process: ChatGPT, Claude AI, and Gemini Pro for polishing ideas and generating code snippets; Grammarly, Code Spell Checker, and LTeX LanguageTool for monitoring linguistic accuracy; and GitHub Copilot for speeding up prototype development. The open-source community and many tool and extension developers significantly increase work efficiency. Diagramming tools aid in creating figures and diagrams; the author used images from the internet in this book from trusted online sources.

Family and friends encourage the author throughout his academic journey, motivating him. This thesis has shown the author the power of collaborative academic research and supportive communities at all primary levels of education.