This thesis developed, implemented, and evaluated an AI-powered, web-based Mini-Mental State Examination (MMSE) application, introducing an innovative approach to cognitive assessment. The research addressed longstanding challenges in neuropsychological testing by integrating advanced AI technologies, such as Large Language Models (LLMs) including Llama 3.1 and ChatGPT 4.0, with traditional assessment methods.
The study's primary findings underscore the significant potential of AI-enhanced cognitive assessment tools:
\begin{enumerate}
\item \textbf{Accuracy and Consistency}: The Llama 3.1:70B model achieved a 92.9 percent success rate in confirming the correctness of responses within a specific test suite, demonstrating its potential to enhance the reliability of cognitive assessments.
\item \textbf{Enhanced Accessibility}: The web-based application transcended geographical barriers, making cognitive assessments more accessible to underserved populations and facilitating remote evaluations.
\item \textbf{Standardization}: Automated scoring and analysis contributed to more consistent assessment processes across varied settings, reducing the variability associated with human scoring.
\item \textbf{Adaptability}: AI models enabled dynamic adjustments in test difficulty and personalization, potentially increasing sensitivity to subtle cognitive changes.
\end{enumerate}
However, several challenges remain:
\begin{enumerate}
\item \textbf{Response Time}: Optimizing the balance between accuracy and speed, particularly with the Llama 3.1 model, is necessary. The current web-based version requires longer administration times compared to traditional methods.
\item \textbf{Data Privacy and Security}: Safeguarding sensitive healthcare data within AI systems is critical and requires robust measures.
\item \textbf{Accessibility for Older Adults}: Bridging the digital divide among older populations is essential for the widespread adoption of this technology.
\item \textbf{Ethical Considerations}: The use of AI in cognitive assessments raises ethical concerns regarding fairness, transparency, and potential bias.
\end{enumerate}
This research contributed significantly to the evolving intersection of AI and healthcare diagnostics. Demonstrating the feasibility of AI-powered cognitive assessments established a foundation for future research and development in this critical area.
Future research should focus on:
\begin{enumerate}
\item \textbf{Larger-Scale Validation}: Conducting studies with more extensive and diverse samples to confirm the effectiveness and reliability of the AI-powered MMSE.
\item \textbf{Longitudinal Studies}: Evaluating the utility of the AI-powered MMSE in tracking cognitive changes over time.
\item \textbf{Cross-Cultural Adaptation}: Expanding support for multiple languages and cultural contexts to enhance global applicability.
\item \textbf{Integration with Health Records}: Exploring seamless integration with electronic health record systems to increase clinical utility.
\item \textbf{Advanced AI Techniques}: Investigating sophisticated AI approaches, such as federated learning, to address data privacy concerns.
\end{enumerate}
In conclusion, this research marked a significant step toward integrating AI into cognitive health assessments. While the findings are preliminary, the potential benefits of enhanced standardization, accessibility, and efficiency are considerable. As AI technologies continue to evolve and ethical and practical concerns are addressed, the future holds promise for improved cognitive assessments, leading to earlier detection and better management of cognitive decline. Continued research and development in this area will likely improve patient outcomes and support healthcare providers in delivering high-quality care.

\subsection*{Acknowledgments}
The author extends sincere gratitude to \supervisordr{} for his expert guidance and invaluable support throughout this research. Appreciation is due to the University of Tartu's Institute of Computer Science for providing an excellent academic environment and resources. Eduardo Brito's advice and shared ideas merit special recognition.

This work benefited from various technical tools and AI assistants, as detailed in Appendix~\ref{appendix:writing-workflow}. PlantUML and draw.io facilitated diagram creation, with draw.io being particularly valuable for more complex and visually detailed diagrams. The author acknowledges the experts who tested the application and provided feedback, contributing significantly to the refinement of the research outcomes.

Gratitude is extended to family and friends for their unwavering support throughout this academic journey. This thesis demonstrates the power of collaborative academic research and supportive communities in education.