The demographic situation in Europe is changing dramatically due to the fast-growing aging population. According to the Eurostat view, the share of those 65 and older increased from 20.3\% in 2019 to 29.4\% by 2050 \cite{eurostat2020}. This important demographic change corresponds to a rise in the incidence of age-related cognitive impairments like MCI and different forms of dementia \cite{livingston2020}. Recent studies further estimate that MCI now prevails in 6-12\% of adults aged 60 and above \cite{petersen2018}. During the same period, the number of patients with dementia will grow from 9.78 million in Europe in 2018 to 18.85 million in 2050 \cite{alzheimer2019}.

This increased prevalence of cognitive impairment comes at a huge cost to health systems, families, and societies. However, the timely detection and identification of interventions have been noticed to make a vast difference in the progression of decline. Such timely identification may facilitate healthcare providers' intervention to slow progress, improve quality of life, and reduce the overall burden on healthcare systems \cite{livingston2020}. Thus, this offers a means whereby timely diagnosis will enable patients and their families to plan a future care pathway, allowing them to decide about the various treatment options brought about by different available avenues \cite{sperling2011}.

Considering these views, an emerging need is pressing in nature for efficient, accessible, and trustworthy cognitive assessment tools, which would enable mass-scale screening along with monitoring of the aging population based on cognitive function. From this, we hasten to give reasons for our research to develop a much-improved Web-based Mini-Mental State Examination system for enhancing accessibility and efficiency in cognitive assessment within clinical and research settings.